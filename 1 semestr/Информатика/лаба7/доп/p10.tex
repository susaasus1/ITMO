\newpage
\rhead{\textbf{\textcolor{blue}{Т}\textcolor{gray}{ерминология: Информация и данные}}}
\makebox[0pt][l]{\includegraphics[scale=0.5]{1} }
\vspace*{4mm}

\textcolor{Green}{Информатика}
– дисциплина, изучающая свойства и структуру информации,
закономерности ее создания, преобразования, накопления, передачи и
использования.

\vspace*{2mm}
\textcolor{Green}{Англ}
: informatics = information technology + computer science + information
theory

\vspace*{5mm}
\begin{center}
\textbf{Важные даты}
\end{center}
$\bullet$  \quad 1956 – появление термина «информатика» (нем. Informatik, \\
\qquad Штейнбух) \\
$\bullet$  \quad  1968 – первое упоминание в СССР (информология, Харкевич) \\
$\bullet$  \quad 197Х – информатика стала отдельной наукой \\
$\bullet$  \quad 4 декабря – день российской информатики

