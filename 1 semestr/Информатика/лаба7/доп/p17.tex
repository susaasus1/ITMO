\newpage
\rhead{\textbf{\textcolor{blue}{А}\textcolor{gray}{нализ свойств меры Хартли}}}
\makebox[0em][l]{\includegraphics[scale=0.5]{1} }
\vspace*{2mm}
\newline
Экспериментатор одновременно подбрасывает монету (М) и кидает игральную кость  (К).
Какое количество информации содержится в эксперименте (Э)?\\
\vspace*{3mm}
\textcolor{Green}{Аддитивность}:\\
\qquad i(Э) = i(М) + i(К) => i(12 исходов) = i (2 исхода) + i (6 исходов): \\
\qquad$log_x 12$ = $log_x 2$ + $log_x 6$\\
\textcolor{Green}{Неотрицательность}:\\
\qquad Функция $log_x N$ неотрицательно при любом x > 1 и $N \geq 1$. \\
\textcolor{Green}{Монотонность}:\\
\qquad С увеличением p(М) или p(К) функция i(Э) монотонно возрастает\\
\textcolor{Green}{Принцип предопределённости}:\\
\qquad При наличии всегда только одного исхода (монета и кость с \\
\qquad магнитом) количество информации равно нулю: $log_x 1 + log_x 1 = 0$.

