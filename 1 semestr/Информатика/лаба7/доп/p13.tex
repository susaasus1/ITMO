\newpage
\rhead{\textbf{\textcolor{blue}{И}\textcolor{gray}{змерение количества информации}}}
\makebox[0em][l]{\includegraphics[scale=0.5]{1} }
\vspace*{2mm}
\newline
\textcolor{Green}{Количество информации $\equiv$ информационная энтропия -} это численная мера непредсказуемости информации. Количество информации в некотором объекте определяется непредсказуемостью состояния, в котором находится этот объект.
\vspace*{2mm}

Пусть i (s) — функция для измерения количеств информации в объекте s, состоящем из n
независимых частей $s_k$
, где k изменяется от 1 до n. Тогда \textcolor{Green}{свойства меры количества
информации} \textbf{i(s)} таковы:
\newline

\ $\bullet$ \quad Неотрицательность: i(s) $\geq$ 0.\\
\ $\bullet$ \quad Принцип предопределённости: если об объекте уже все\\
\ \qquad   известно, то i(s) = 0.\\
\ $\bullet$  \quad Аддитивность: i(s) = $\sum$ i($s_k$) по всем k.\\
\ $\bullet$  \quad Монотонность: i(s) монотонна при монотонном изменении \\
\ \qquad вероятностей.\\



