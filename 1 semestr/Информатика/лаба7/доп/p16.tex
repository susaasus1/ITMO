\newpage
\rhead{\textbf{\textcolor{blue}{П}\textcolor{gray}{ример применения меры Хартли на практике}}}
\makebox[0em][l]{\includegraphics[scale=0.5]{1} }
\vspace*{2mm}
\newline
\textbf{Пример 1.} Ведущий загадывает число от 1 до 64. Какое количество вопросов типа
«да-нет» понадобится, чтобы гарантировано угадать число?\\
\quad{$\bullet$} \ \underline{Первый} вопрос: «Загаданное число меньше 32?». Ответ: «Да».\\
\quad{$\bullet$} \ \underline{Второй} вопрос: «Загаданное число меньше 16?». Ответ: «Нет».\\
\quad{$\bullet$} \qquad ... \\
\quad{$\bullet$} \ \underline{Шестой} вопрос (в худшем случае) точно приведёт к верному ответу.\\

\quad{$\bullet$} \  Значит, в соответствии с мерой Хартли в загадке ведущего\\
\quad \quad \ содержится ровно $log_2 64$= 6 бит непредсказуемости (т. е. инф-ии).
\newline
\textbf{Пример 2.} Ведущий держит за спиной ферзя и собирается поставить его 
на произвольную клетку доски. Насколько непредсказуемо его решение?\\
\quad{$\bullet$} \ Всего на доске 8х8 клеток, а цвет ферзя может быть белым или  \\
\quad \quad \ чёрным, т.е. всего возможно 8х8х2 = 128 равновероятных состояний.\\
\quad{$\bullet$} \ Значит, количество информации по Хартли равно $log_2 128$ = 7 бит \\



